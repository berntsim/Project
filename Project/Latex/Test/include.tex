%Vi starter med preamble, hvor vi definerer pakker osv som vi bruker i dokumentet. Hvis ikke du vet hva du gjør, så ikke endre noe på dette.

%%PREAMBLE
%%===================================
\documentclass[12pt, a4paper]{article}
\usepackage{graphicx}
%\usepackage[english]{babel}
\usepackage[utf8x]{inputenc}
\usepackage{color}
\usepackage{multirow}
%\usepackage[margin=1in]{geometry} %BESTEMMER STORRELSEN PA DOKUMENTETS MARG.
\usepackage{wrapfig}
\usepackage{subcaption}
\usepackage{hyperref}
\usepackage{amsmath}
\usepackage{amssymb}
\hypersetup{
    colorlinks,
    citecolor=black,
    filecolor=black,
    linkcolor=black,
    urlcolor=black
}
\usepackage{titlesec}
\usepackage{tabularx}
\usepackage[parfill]{parskip}
\usepackage{array}
\newcolumntype{L}[1]{>{\raggedright\let\newline\\\arraybackslash\hspace{0pt}}m{#1}}
\renewcommand{\arraystretch}{1.5} % increase vertical spacing in table cells

\usepackage[nottoc,notlot,notlof,numbib]{tocbibind}

\setcounter{tocdepth}{2}

\renewcommand*\thesection{\arabic{section}}

%\usepackage[backend=bibtex]{biblatex}
%%===================================

%I seksjonen under begynner selve dokumentet. Det er organisert slik at man skriver hver seksjon som eksterne dokumenter, og deretter inkluderer dem inn i hoveddokumentet. Dette gjøres ved å skrive \inlcude{ditt dokuments navn}.

%Vi bestemmer også stilen på dokumentet her. Dette er i utgangspunktet mulig å endre en del på, men det kan være stress, og ting kan fuckes opp ganske uten varsel, så anbefaler å være forsiktig.

%\titleformat{\chapter}{\normalfont\huge}{\thechapter.}{20pt}{\huge\textbf}

\newcommand{\MarginText}[1]{\marginpar{\raggedleft\small#1}}
\reversemarginpar

%\usepackage{refcheck}

%%===================================

