\documentclass{article}
\usepackage[utf8]{inputenc}
\usepackage[]{graphicx}
\usepackage{listings}
\usepackage[toc,page]{appendix}
\usepackage{float}

\title{Numerical project on iterative maps}
\author{Simen Berntsen}


\begin{document}
\maketitle



%%%%%%%%%%%%%%%%%%%%%%%%%%%%%%%%%%%%%%%%%%%%%%%%%%%%%%%%%%%%%%%%%%%
\section{Introduction}
In this numerical project, a simulation of the Henon map, and the Peter de Jong-attractor was done using python. The python program does the iterations according to what is specified in the exercise text, and plots the results. Fir a complete description, see the assignment text \cite{ex_text}

%%%%%%%%%%%%%%%%%%%%%%%%%%%%%%%%%%%%%%%%%%%%%%%%%%%%%%%%%%%%%%%%%%%
\section{Theory}

The Henon map is an iterative map given by the following equations:

\begin{equation}
x_{n+1} = x_{n}\cos(a) - (y_{n} - x_n^2)\sin(a)
\end{equation}
\begin{equation}
y_{n+1} = x_{n}\sin(a) + (y_{n} - x_n^2)\cos(a)
\end{equation}

The Peter de Jong-attractor is given by:

\begin{equation}
x_{n+1} = \sin(ay_n) - \cos(bx_n)
\end{equation}
\begin{equation}
y_{n+1} = \sin(cy_n) - \cos(dx_n)
\end{equation}

%%%%%%%%%%%%%%%%%%%%%%%%%%%%%%%%%%%%%%%%%%%%%%%%%%%%%%%%%%%%%%%%%%%
\section{Results}
\subsection{Henon map}
In this section, the resulting scatter plots from the simulations is presented. The specifics of the assignment will not be stated, but the reader is reffered the the assignment text \cite{ex_text}. To begin with, the Henon map is analyzed for $ a = \pi/2$, since this is well within the suggested interval. The resulting plot is found in figure 




To have something to compare this to, the Henon map was also analyzed for $a = 3$. The corresponding plot is found in figure \ref{fig:henon_3}. Comparing the two, one can see that the $a = \pi/2$ plot has more points "out of order", i.e. it seems more unstable than the plot of $a = 3$. For the latter, the points seems ordered around the center, with hardly any points wondering off to infinity.



\subsection{Peter de Jong-attractor}

For the de Jong-attractor, one can obtain even more interesting behaviors than for the Henon map. The plot of the "default" values stated in the assignment text is shown in figure \ref{fig:deJong_1}.


When varying the parameters for the attractor, one could see any number of interesting plots. Keeping task 3 in the exercise in mind, the other plot presented includes an area which is frequently visited (usefull for task 3). This can be found in figure \ref{fig:deJong_2}. It is also interesting that this plot looks almost three dimensional, which is a cool feature. During the experiments with varying the parameters, even more 3-D looking maps occured, which was interesting. 


Figure \ref{fig:deJong2_1} shows the resulting plot of using the part of the script designed for task 3, with the parameters and interval given in the assignment text \cite{ex_text}.


More interestingly, from the plot of the de Jong attractor for $a = 0.28, b = -2.4, c = 2.4, d = -2.8$, one can see that there is a cluster of points around   $ x \in [1.25,1.3], y \in [1.0,1.1]$. This may then be used as a start point for another analysis. The result can be found in figure \ref{fig:deJong2_2}. 


What one can see from these plots is that the points follow (as one would expect) the points of the plots from which they were "seeded". However, the behavior of these points are chaotic, due to the wide spread of point, even though the starting points are fairly close to one another. This is more so in figure \ref{fig:deJong2_2}, since the interval of $x,y$ is larger here. This means that there might be a difference in the degree of chaos, but the behavior is chaotic nonetheless.


%%%%%%%%%%%%%%%%%%%%%%%%%%%%%%%%%%%%%%%%%%%%%%%%%%%%%%%%%%%%%%%%%%%
\begin{thebibliography}{99}
\bibitem{ex_text}
Ervik, A., NUMERICAL ASSIGNMENT, Department of physics, NTNU, 2011.
\end{thebibliography}






\end{document}








