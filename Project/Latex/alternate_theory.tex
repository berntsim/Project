\section{Theory}

\subsection{Fractal dimension}
To compare the DLA simulations to experimentally observed clusters, one studied and compared the density correlations (and fractal dimensions). The general expression for correlations are 

\begin{align}
C(\textbf{r}) = \langle \rho(\textbf{r'})\rho(\textbf{r'} +\textbf{r}) \rangle.
\label{eq:general_corr}
\end{align}

In equation \eqref{eq:general_corr}, $C(\textbf{r})$ is the correlation function, $\rho$ is the density and $\textbf{r}$ and $\textbf{r'}$ are points in space. For particle clusters such as those generated by DLA, one defines this density to be $\rho(\textbf{r}) = 1$ for occupied points, and $0$ otherwise. An approximation of the ensemble average of the density correlation function which is valid for $ r_{cluster} \gg r_{particle}$ is given  as (from \cite{PhysRevLett.47.1400})

\begin{align}
C(\textbf{r}) = \frac{1}{N}\sum_{\textbf{r'}}\rho(\textbf{r'})\rho(\textbf{r'} + \textbf{r}).
\label{eq:approx_corr}
\end{align} 

In equation \eqref{eq:approx_corr}, $N$ is the number of particles in the cluster. All other quantities are as in \eqref{eq:general_corr}. It can then be shown that $C(r)$ has the scaling relation \textcolor{red}{Some papers just state that the general scaling relation is assumed to be of the form... should look into this more closely}

\begin{align}
C(r) \sim r^{\alpha}.
\end{align}

Moreover, the mass of the cluster scales like
\begin{align}
M(r) \sim r^{d_f},
\end{align}

so that 
\begin{align}
C(r) = \frac{dM(r)}{2\pi r dr} \sim \frac{r^{d_f-1}}{2\pi r dr} \sim r^{d_f-2}.
\label{eq:d_f-2}
\end{align}

from equation \eqref{eq:d_f-2}, one may generalize the equation by replacing the $2$ in the last exponent with $d$ (the euclidean dimension), so that comparison of exponents yields 

\begin{align}
d_f = d - \alpha.
\label{eq:d_f corr}
\end{align}

\textcolor{red}{??? should include how to calculate the fractal dimension from density correlation function!}

An alternative way of calculating the fractal dimension is through the radius of gyration $R_g$. This stems from the scaling relation 

\begin{align}
N = k_0\left(\frac{R_g}{a}\right)^{d_f},
\label{eq:N_Rg_rel}
\end{align}

where $N$ is the number of particles in the cluster, $k_0$ is the fractal prefactor (also called structural coefficient), $a$ is the diameter of the particles, and $d_f$ is the fractal dimension. In equation \eqref{eq:N_Rg_rel}, $R_g$ is defined as 

\begin{align}
R_g = \frac{1}{N} \sum_{i=1}^{N} r_i, 
\end{align}

where $r_i$ is the distance from particle number $i$ to the centre of mass $R_{CM}$. \textcolor{red}{Mention the substitution $M \leftrightarrow N$ because of $m_i = 1$. }

\subsection{Family-Vicsek scaling}
A scaling relation for the cluster size distribution $n_s(t)$ has been found by F. Family and T. Vicsek \cite{PhysRevLett.52.1669}. \textcolor{red}{??? Write about scaling relation, and how this is used to analyze moving clusters.} The distribution is defined as $n_s(t) = N_s(t) / L^2$, where $N_s(t)$ is the number of clusters containing $s$ particles after time $t$. $L$ is the lattice size \textcolor{red}{(???$L^2$?)}. One should point out the importance of how time scales in the simulation, as this is a numerical experiment. It has been found that $n_s(t)$ scales as a power-law

\begin{align}
n_s(t) \sim t^{-w}s^{-\tau}f\left(\frac{s}{t^z}\right),
\end{align}
where $w$, $\tau$ and $z$ are fitting \textcolor{red}{(??? not sure if correct word)} parameters. 