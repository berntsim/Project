\section{Introduction}

This project has been on taking a Diffusion-Limited Aggregation (DLA) approach to the agglomeration of aerosols in the lower atmosphere in relation to emission from fossil fuelled based combustion (etc.) causing emission of dry carbon. Diffusion-Limited Aggregation is an algorithmic approach to a growth process where limited supply of mass is the limiting factor. A number of different growth models have been proposed, however the original one was proposed by T. A. Witten Jr. and L. M. Sanders \cite{PhysRevLett.47.1400}. This model builds on the Eden model, first proposed by M. Eden \cite{eden1961}, but with a random walker determining the aggregation.  A more dynamic model for agglomeration of moving clusters was proposed by P. Meakin \cite{PhysRevLett.51.1119}. This model was analysed using the scaling relation suggested by T. Vicsek and F. Family \cite{PhysRevLett.52.1669}. The goal of this project has been to use these methods, particularly the diffusion limited cluster aggregation (DLCA), to model the agglomeration of carbon immediately after emission, and to compare the results to what is experimentally observed by Scarnato et al. \cite{acpd-15-2487-2015}. \textcolor{red}{This should be more tied together! Make the point that it has not been to include the mineral dust into the model, but to recreate the carbon clusters. Should also be revised when all the "work" is done, to see what was accomplished, and not promise things you don't deliver.}