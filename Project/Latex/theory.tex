\section{Theory}

\subsection{Fractal dimension}
The clusters are analysed according to their fractal dimension (Fraunhofer dimension), defined by their radius of gyration according to the following formula.

\begin{align}
N = k_0\left(\frac{R_g}{a}\right)^{d_f},
\label{eq:N_Rg_rel}
\end{align}

where $N$ is the number of particles in the cluster, $k_0$ is the fractal prefactor (also called structural coefficient), $a$ is the diameter of the particles, and $d_f$ is the fractal dimension. The radius of gyration, $R_g$, is defined as 

\begin{align}
R_g = \frac{1}{N} \sum_{i=1}^{N} r_i, 
\end{align}

where $r_i$ is the distance from particle number $i$ to the centre of mass $R_{CM}$. \textcolor{red}{Mention the substitution $M \leftrightarrow N$ because of $m_i = 1$. }

\subsection{Family-Vicsek scaling}
A scaling relation for the cluster size distribution $n_s(t)$ has been found by F. Family and T. Vicsek \cite{PhysRevLett.52.1669}. \textcolor{red}{??? Write about scaling relation, and how this is used to analyze moving clusters.} The distribution is defined as $n_s(t) = N_s(t) / L^2$, where $N_s(t)$ is the number of clusters containing $s$ particles after time $t$. $L$ is the lattice size \textcolor{red}{(???$L^2$?)}. It has been found that $n_s(t)$ scales as a power-law

\begin{align}
n_s(t) \sim t^{-w}s^{-\tau}f\left(\frac{s}{t^z}\right),
\end{align}
where $w$, $\tau$ and $z$ are fitting \textcolor{red}{(??? not sure if correct word)} parameters. One should point out the importance of how time and step length scales in the simulation, as this is a numerical experiment. This is due to diffusion and random walks being related through

\begin{align}
D = \frac{(\Delta x)^2}{2\Delta t}.
\label{eq:diffusion_walker_relation}
\end{align}
Here $D$ is the diffusion constant, $\Delta x$ is the step size og the random walker, and $\Delta t$ is the time step. 